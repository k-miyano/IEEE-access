\documentclass{ieeeaccess}
%
% If IEEEtran.cls has not been installed into the LaTeX system files,
% manually specify the path to it like:
% \documentclass[journal]{../sty/IEEEtran}
\usepackage[dvipdfmx]{graphicx}
\usepackage{mathrsfs}
\usepackage{mathtools}
\usepackage{comment}
\usepackage{caption}
\usepackage{graphicx}
\usepackage{amsmath,amssymb,,amsfonts}
\usepackage{ascmac}
\usepackage{bm}
\usepackage{cite}
\usepackage{nidanfloat}
\usepackage{url}
\usepackage{color}
\usepackage{algorithmic}
\usepackage{textcomp}

\usepackage[utf8]{inputenc}
\usepackage[whole]{bxcjkjatype}
\usepackage{xcolor}

\def\rnum#1{\expandafter{\romannumeral #1}}

\def\BibTeX{{\rm B\kern-.05em{\sc i\kern-.025em b}\kern-.08em
    T\kern-.1667em\lower.7ex\hbox{E}\kern-.125emX}}

\newcommand{\argmin}{\mathop{\rm arg~min}\limits}
\newcommand{\argmax}{\mathop{\rm arg~max}\limits}


\begin{document}
\history{Date of publication xxxx 00, 0000, date of current version xxxx 00, 0000.}
\doi{10.1109/ACCESS.2017.DOI}
\title{Utility Based Scheduling for Multi-UAV Search System in Disaster Areas}
%
%
% author names and IEEE memberships
% note positions of commas and nonbreaking spaces ( ~ ) LaTeX will not break
% a structure at a ~ so this keeps an author's name from being broken across
% two lines.
% use \thanks{} to gain access to the first footnote area
% a separate \thanks must be used for each paragraph as LaTeX2e's \thanks
% was not built to handle multiple paragraphs
%

%\author{Michael~Shell,~\IEEEmembership{Member,~IEEE,}
%        John~Doe,~\IEEEmembership{Fellow,~OSA,}
%        and~Jane~Doe,~\IEEEmembership{Life~Fellow,~IEEE}% <-this % stops a space
\author{\uppercase{Kosei MIYANO,Ryoichi SHINKUMA,Eiji OKI, and Takehiro SATO}}
\address{Graduate School of Informatics, Kyoto University Yoshidahon-machi, Sakyo-ku, Kyoto 606-8501 Japan}
\tfootnote{This work was partly supported by JSPS KAKENHI Grant Number JP17H01732.
  A part of the content in this paper was published in the proceedings of IEEE GCCE 2018.
}

\markboth
{K. Miyano \headeretal: Utility Based Scheduling for Multi-UAV Search System in Disaster Areas}
{K. Miyano \headeretal: Utility Based Scheduling for Multi-UAV Search System in Disaster Areas}

\corresp{Corresponding author: K. Miyano (e-mail: kmiyano@icn.cce.i.kyoto-u.ac.jp).}

\begin{IEEEkeywords}
uav (unmanned aerial vehicle), target search, scheduling, edge computing
\end{IEEEkeywords}

\begin{abstract}
%The number of deaths and missing people due to natural disasters is still a big social issue these days in many countries.
%It is a still big social issue these days to find people who have disappeared because of getting lost in unexpected situation like disaster.
%To search such lost people in disaster-damaged areas, the integrated system of the mobility of multiple unmanned aerial vehicles (UAVs) has been studied as a promising solution for that.
Micro or small unmanned aerial vehicles (UAVs) is a promising solution for finding people who have disappeared because of getting lost in unexpected situation like disaster.
It takes long time to analyze the acquired image data for the target recognition due to the limited computational resource of small UAVs.
In addition, the data transfer time can increase in the disaster-hit areas due to the  damage of communication infrastructures.
However, the prior researches did not consider both processing time of the acquired data and data transfer time, despite the temporal requirement in the surveillance scenarios.
Therefore, this paper proposes a scheduling method of multi-UAV search system that considers processing time of image data and data transfer time.
We present the utility-based problem formulation that ensures the freshness of individual obtained piece of information while obtaining as many pieces of information as possible for a certain period.
Simulation results verify that the proposed scheduling method ensures the freshness of individual obtained piece of information while delivering as many pieces of information as possible for a certain period by evaluating each information in terms of two metrics: i) elapsed time from start time and ii) elapsed time after acquired.
\end{abstract}

\maketitle
\IEEEpeerreviewmaketitle


%\vspace*{-3mm}

\section{introduction}\label{intro}
The number of deaths and missing people due to natural disasters is still a serious problem in many countries.
According to a report by Centre for Research on the Epidemiology of Disasters\cite{CRED2016}, the average number of deaths and missing people due to natural disasters occurred all over the world, such as earthquakes, hurricanes, forest fires, and floods, from 2006 to 2015 was approximately 70,000.
In order to reduce the number, one of the solutions to reduce the number is to increase the number of rescue teams.
The report by Japan Ministry of Defense after the great east Japan earthquake suggests that  it is necessary to secure manpower through guidelines for the concentration of units in the immediate aftermath of a disaster.
%
However, a huge budget is required to secure manpower and the risk of secondary damages in disaster occurrence areas is serious remaining issues to be solved\cite{disaster2011}.
%
Micro or small unmanned aerial vehicles (UAVs), also known as
drones, are expected to be emerging solutions to solve the above problem in areas where humans and ground vehicles cannot easily step into like disaster-damaged areas.
Technological advances in the recent years have led to the emergence of smaller and cheaper UAVs, which have some functions such as transporting relief supplies, collecting data by using equipped sensors and operating as adhoc wireless mesh network infrastructure in such isolated areas \cite{Andre2014,Erdelj2016,Felice2014}.
Collecting image data is especially important because it is applicable to many use cases such as search and rescue mission, fire detection and surveilance.
%
Since  there are time constraints in such situations, UAVs need to collect sensor data as soon as possible.
In the great east Japan earthquake, as time passed, the number of deaths and missing people  dramatically increased\cite{japan2011}.
Surveillance using multiple UAVs has been receiving increasing attention for reasons such as increase in system reliability, robustness, and efficiency\cite{Lanillos2014,Maza2007,Meng2014,chang2016,Mirzaei2011}.
However, despite the temporal requirement in these surveillance systems, it is assumed that the user can obtain necessary information as soon as UAV acquires image data: both processing time of image data and data transfer time are not taken into consideration. In the practical situation, it takes long time to analyze images for target recognition due to the limited computational resource of small UAVs. In addition, the data transfer time can increase in the disaster-hit areas due to the damage of communication infrastructures.


Therefore, this paper proposes a scheduling method of multi-UAV search system that considers processing time of image data and data transfer time.
We present the problem formulation of the proposed scheduling method that maximizes the user’s utility, which is calculated from the efficiency of obtaining results from analyzed data and the interval of obtaining the results.
We show the results of performance evaluation to verify that the proposed method ensures the freshness of individual obtained piece of information while delivering as many pieces of information as possible for a certain period.

The remainder of this paper is organized as follows. Section II discusses the related work. Section III presents the system overview and proposed scheduling method. Section IV then provides the performance evaluation of the proposed method through a simulation, followed by the extension to two-dimensional model in Section V. Finally, Section VI concludes this paper.
%
%
%
%
%
%
%
%
%
%
%
%
\section{Related work}
This section discusses the prior works related to this paper.
First we will discuss UAV Applications in disaster areas. 
Among the application scenarios that will be introduced in Section \ref{app}, data collection using equipped sensors is carried out in our system.
Next the prior researches on Multi-UAV cooperation for area coverage is discussed since it is also assumed in our search system as mentioned in Section \ref{intro}.
Finally, we also discuss prior works on inter UAV cooperation for computing and  computing on UAVs with edge computing.
In our system, it is assumed that the collected data is processed locally onboard UAVs with edge computing but inter UAV cooperation for computing is not considered since the overhead of transmission between UAVs is large and is out of the scope of this paper.

\subsection{UAV Applications in disaster areas}\label{app}
Transporting relief supplies by UAVs is very important since there is a possibility that humans and ground vehicles cannot easily step into the disaster areas.
Bamburry mentioned the ability of a UAV to deliver medical products to remote and hard-to-reach areas\cite{Bamburry2015}.
For example, in the devastating 2010 earthquake in Haiti, a UAV delivery system was used to deliver medicine to camps set up after the disaster\cite{May2015}.

UAVs can also collect data by using equipped sensors. In \cite{Wada2015}, Each UAV is provided with a mobile optical sensors and image transmission modules developed by the Wada et al. The optical sensor which is a combination of a IR sensor and a visible-light sensor enables data collection even at night or against smoke in the disaster areas.
After the launch, a UAV executes auto flight along the way points by recognizing its positions and  obtains the necessary video/image information. The UAV transmits it to the server and shares it with users via the Internet.

UAVs also often operate as adhoc wireless mesh network infrastructure, which is called Flying Ad Hoc Networks (FANET)\cite{Bekmezci2013}.
In 2016, S\'anchez et al. aim to provide connectivity for rescuers and disaster victims using UAVs\cite{Garcia2016}.
They propose a Jaccard-based movement rules to define the UAVs best positions for providing the best communication service to the victims in a urban disaster scenario.
Finally they compare among several local search computational intelligence algorithms implemented such as simulated annealing, hill climbing, and random walk for deciding the best tactical UAV movements.

\subsection{Multi-UAV cooperation for area coverage}\label{cover}
Maza et al. provided a pioneer work in cooperatively searching a given area to detect objects of interest by UAVs\cite{Maza2007}.
First they determine relative capabilities of each UAV, based on factors like flight speed, altitude required for the mission, sensitivity to wind conditions and sensing width.
Then, they divide the whole area by divide-and-conquer, taking into account the UAV’s relative capabilities and initial locations.
Finally, they set the waypoints of each UAV so that the number of turns needed along a zigzag pattern is minimized.

Zhao et al. in 2016 tackled the challenging problem of not only searching the target area for a lost target but also tracking the target\cite{chang2016}.
In the tracking stage, each UAV keeps desired distance with the target, coordinating the angular separation between neighboring UAVs to the same angle.
if there is a shelter between UAV and target, the target state is predicted by the target model with the former target information.  
In the searching stage, multi-UAVs divide the search region equally which is determined by the target lost duration time and speed and then search for the target by the method of shrinking annulus.
The switch tactics between the tracking stage and the searching stage was also proposed. 

In 2017, Hayat et al. proposed a multi-objective optimization algorithm to search and plan paths for UAVs\cite{Hayat 2017}.
UAVs search for the target cooperatively and soon after some UAV detects the target, the other UAVs takes positions for relay chain formation between the UAV and a base station.
The algorithm aims to minimize the mission completion time, which includes the time to find the target and the time to setup a communication path.
Finally they compare among three strategies that perform search by UAVs in a similar manner but have a different path planning in terms of the mission completion time.

\subsection{Multi-UAV cooperation for computing}\label{compute}
UAV Applications in disaster areas require UAVs to deal with intensive computation tasks such as image/video processing, pattern recognition and feature extraction. 
Computation offloading is very important since computational power of a single UAV is limited.
\subsubsection{inter UAV cooperation}
Ouahouah et al. in 2017 proposed the use of offloading mechanism among UAVs equipped with IoT devices\cite{Ouahouah2017}.
Each IoT task is partitioned into a set of sub-tasks that can be executed simultaneously among a cluster of UAVs.
The sub-tasks is assigned to UAVs based on their power supply, resources in terms of memory and CPU computation, and their on-board IoT devices.
Two solutions were proposed for  computation offloading:Energy aware optimal task offloading and Delay aware optimal task offloading.
The former maximizes the UAVs lifetime by electing the UAVs with higher power supply.
The latter reduces the response time by favoring the selection of UAVs with more resource capacities.

In 2018, Valentino et al. proposed an opportunistic and adaptive computational offloading scheme between UAV clusters\cite{Valentino2018}.
A cluster head will broadcast a ‘hello’ message indicating their presence and available resources and then a local cluster send an offloading request to a desired cluster head.
a local cluster decides if it is better to do the task alone or to offload, estimating response time for doing the computational offloading and processing the given task through computing power, size of task, bandwidth, and data rate of wireless network.

\subsubsection{Edge computing}
Edge computing has been proposed as an effective mean of supplementing computational resources for UAVs\cite{Motlagh2017,Messous2017}.

Motlagh et al. in 2017 demonstrated how UAVs can be used for crowd surveillance based on face recognition. 
Due to the computational overhead required by such a use case and given the limited power supply of UAVs, they performed the offloading of video data processing to a mobile edge computing node.
The obtained results showed clearly the benefits of computation offloading compared to the local processing of video data onboard UAVs in saving energy and quickly detecting and recognizing suspicious persons in a crowd.

Messous et al. in 2017 tackled a computation offloading problem with three different devices: UAV, base station and edge server, which carry out the heavy computation tasks.
They prove the existence of a Nash Equilibrium and design an offloading algorithm that converges to the optimal point.
Their cost function was defined as a combination of two performance metrics: energy and delay.
They finally achieved better value of the utility using the the offloading algorithm, compared to computing on: edge server, base station and drone respectively.
%
%


\section{Proposed method}\label{method}

\subsection{System overview}\label{sys}
\subsubsection{System model}\label{sysmo}

\begin{figure}[htbp]
\begin{center}
\includegraphics[width=8.0cm]{fig/system_illustration.pdf}
%\vspace{-2.5mm}
\caption{System illustration}
\label{model}
\end{center}
\end{figure}

Figure \ref{model} depicts the system model we assume in this paper. 
The system consists of a user device (UD), multiple UAVs, and an edge server (ES), the roles of which are described as below.
%
The UD is the central operating entity in the system and operates all the UAVs and the ES; the UD determines flying routes and timings of UAVs and assigns workloads of computing sensor data to UAVs and the ES.
%
The UD also works to forward sensor data received from UAVs to the ES and to obtain computational results from both UAVs and the ES.

Each UAV is operated by the UD and performs the following actions autonomously in the distributed manner.
%
(I) Flying between the initial position, at which the UAV can communicate directly with the UD, and the sensing region assigned to the UAV 
(I\hspace{-.1em}I) Acquiring image data (still or moving images) of the sensing region assigned to the UAV.
(I\hspace{-.1em}I\hspace{-.1em}I) Staying at the initial position and performing the following actions in parallel: (a) analyzing a part of collected image data with the computational power of the UAV and (b) delegating the analysis of the rest of the data to the ES.
(I\hspace{-.1em}V) Reporting results obtained from the analysis of image data to the UD soon after it has been completed.
%
Note that we assume that UAVs cannot perform any analysis while flying; computational resources of UAVs are fully used for flight control and image acquiring during the flight.Each UAV repeats all the actions (I) to (I\hspace{-.1em}V). We call one action (I) to (I\hspace{-.1em}V) of some UAV one round.

The ES is placed closely to the UD and works to perform the analysis of a part of image data received from UAVs.
%
Like UAVs do, soon after the ES has completed the analysis, it reports the results to the UD.
%
%Edge computing has been proposed as an effective mean of supplementing computational resources for UAVs because, in general, computational power of small or micro UAVs is limited \cite{Mohamed2017,Motlagh2017}.

\subsubsection{System flow}\label{flow}
In the system we assume in this paper, the schedules of flying, acquiring, and analyzing of UAVs are determined through the following steps:
%
\begin{description}
\item[(1)] Check if there is one or more UAVs the schedules of which have not been determined yet.
%\item[(2)] If step (1) is yes, one of the unscheduled UAVs is picked and labeled as UAV $i$ $i (i=1, 2, 3\cdots)$.
\item[(2)] Pick one of the unscheduled UAVs and label it as UAV $i (i=1, 2, 3\cdots)$, if step (1) is yes.
\item[(3)] Refer to the information about the schedules of UAVs $i-1$, $i-2$, $i-3$, $\cdots$, which were scheduled before UAV $i$.
\item[(4)] Determine the schedule of UAV $i$ by a scheduling method, which will be described later.
\item[(5a)] Increment $i$ to $i+1$. Go back to step (1).
\item[(5b)] UAV $i$ starts its operation based on the determined schedule and will be added to the list of unscheduled UAVs after completing all the actions (I) to (I\hspace{-.1em}V) mentioned in Section \ref{sysmo}.
\end{description}
%
Note that (5a) and (5b) are executed in parallel.
%
At step (4), the scheduling method requires some time for calculating the schedule of UAV $i$.
%
The calculating time depends on the complexity of the scheduling method.
%
Therefore, the complexity of the scheduling method should not be complicated.
%
However, from the second round of the scheduling for a UAV, the calculation can be done in advance while the UAV is performing step (5b) because all the previous schedules before the UAV have been determined already and the information of all the previous schedules are available.
%
%
When the ES receives a computational task from a UAV via the UD, the ES puts it to the waiting queue.
%
The ES processes those computational tasks in the first-in first-out (FIFO) manner and reports the result to the UD immediately after finishing each computational task.

\subsection{Proposed scheduling method}\label{math}
We assume the system model shown in Figure  \ref{model1.5}, in which sensing sections are placed on the one-dimensional line and their sizes are identical.
%
Sensing sections are assigned to UAVs from the one closest to the initial position and only an image is acquired at each sensing section.
%
These assumptions allow us to simply deal with the sensing range of each UAV as the number of images acquired by them.
%
We also assume that, if the computing resource of ES or the communication channel of UD is still used by previously scheduled UAVs (UAVs $1, 2, \cdots i-1$), UAV $i$ has to wait in the first-in first-out (FIFO) manner until their operations are completed.
%
This also means that the operation of UAV $i$ does not affect the operations of the previous UAVs (UAVs $1, 2, \cdots i-1$).

\begin{figure}[t]
\begin{center}
\includegraphics[width=7.0cm]{fig/onedimention.pdf}
\caption{System model for problem formulation}
\label{model1.5}
\end{center}
\end{figure}

\subsubsection{Utility}\label{to}
This section presents the mathematical formula of the utility function.
%
The utility function of UAV $i$ in the proposed method, $U^i$, is given as:

\begin{align}
U^i = \frac{\eta^i}{{\Delta{t}}^i}, \label{ut}
\end{align}
where $\eta^i$ means the efficiency of obtaining results from analyzed data and ${\Delta{t}}^i$ is the interval of obtaining the results.
%
They are called acquisition efficiency and acquisition interval, respectively, and defined as:

\begin{align}
\eta^i&=\frac{N^i}{{t_{fin}^i}-{t_{start}^i}} \label{f1}\\
{\Delta{t}}^i &= {t_{fin}^i}-t_{fin}^{i-1}~~~~(t_ {fin}^0=0, {t_{fin}^i}\geq{t_{fin}^{i-1}}), \label{f2}
\end{align}
where $N^i$ means the number of images acquired by UAV $i$, $t_{start}^i$ is the flight start time of UAV $i$, and $t_{fin}^i$ is the time when processing $N^i$ images is finished.
%
The utility function in defined by (\ref{ut}) suggests that, as the acquisition efficiency and interval become higher and shorter, the system works better for users.
%
The reason why it is reasonable is because, in surveillance scenarios, users would expect to obtain as many pieces of information as possible during a certain period, while more updated information would be more valuable for them.


\subsubsection{Problem formulation}

\begin{table}[t]
\centering
%{\renewcommand\arraystretch{1}
\caption{Definition of notation}
  \begin{tabular}{|c|p{6cm}|} \hline
 & Description
 \\ \hline
 $N^i$ & No. of images acquired by UAV $i$ \\ \hline
 $N_u^i$ & No. of images processed by UAV $i$ in $N^i$ \\ \hline
 $N_e^i$ & No. of images delegated from UAV $i$ to ES   \\ \hline
 $T_{ng}$ & Flying time per sensing section  \\ \hline
 $T_g$ & Image acquisition time per sensing section \\ \hline
 $T_f$ & Flying time from initial position to left-end of sensing sections \\ \hline
 $T_{u,d}^{i}(N^i,N_u^i)$ & Transmission waiting time from UAV $i$ to UD  \\ \hline
 $T_{d,e}^{i}(N^i,N_u^i)$ & Transmission waiting time of UAV $i$'s data from UD to ES  \\ \hline
 $T_e^{i}(N^i,N_u^i)$ & Processing waiting time at ES  \\ \hline
 $\mu_{u,d}$ & Transmission speed from UAV $i$ to UD in no. of images per unit time \\ \hline
 $\mu_{d,e}$ & Transmission speed from UD to ES in no. of images per unit time \\ \hline
 $P_u^i$ & Processing speed at UAV $i$ in no. of images per unit time \\ \hline
 $P_e$ &  Processing speed at ES in no. of images per unit time \\ \hline
\end{tabular}
\label{para}
\end{table}

This section discusses the problem formulation of the proposed scheduling method.
%
Table \ref{para} lists the definition of the notation we use.
%
In this table, $N^i$, $N_u^i$, and $N_e^i$ are variable.
%
Using the notation, the problem formulation is described as below:

\begin{align}
\argmax_{N^i,N_u^i} \quad&  U_i =\frac{\eta^{i}}{{\Delta{t}}^i}  = \frac{N^i / (t_{fin}^i(N^i,N_u^i)-t_{start}^i)}{t_{fin}^i(N^i,N_u^i)-t_{fin}^{i-1}}\label{eq1}\\
s.t. \quad& {N_{MIN}^i}\leq{N^i}, \label{eq2}
\end{align}
where $N_{MIN}^i$ means that UAV $i$ has to acquire at least $N_{MIN}^i$ images so as not to complete its actions earlier than the previous UAV, UAV $i-1$: ${t_{fin}^i}\geq{t_{fin}^{i-1}}$.
This formulation suggests that $N^i$ and $N_u^i$ must be determined so that the utility function, $U^i$, is maximized.

$t_{fin}^i$ in (\ref{eq1}) is represented as:
\begin{align}
&t_{fin}^i(N^i,N_u^i)=t_{start}^i+2(T_f+\sum_{j=1}^{i-1}{{N^j}T_{ng}})+\nonumber\\
&\hspace{-1mm}N^i({T_g+T_{ng}})+\max(\frac{N_u^i}{P_u^i},T_{u,d}^{i}(N^i,N_u^i)+\nonumber\\
&\hspace{-1mm}\frac{N_e^i}{\mu_{u,d}}+T_{d,e}^{i}(N^i,N_u^i)+\hspace{-1mm}\frac{N_e^i}{\mu_{d,e}}+T_{e}^{i}(N^i,N_u^i)+\hspace{-1mm}\frac{N_e^i}{P_e}). \label{eq_fin}
\end{align}

Eliminating $N_e^i$ from (\ref{eq_fin}) using $N^i=N_u^i+N_e^i$, $t_{fin}^i(N^i,N_u^i)$ becomes a function of two variables, $N^i$ and $N_u^i$.
%
According to (\ref{eq_fin}), $t_{fin}^i-t_{start}^i$ is equal to the sum of flight time, transmission time, and processing time of UAV $i$.
%
The second and third terms in the right side of (\ref{eq_fin}) represent the flight time outside the sensing range of UAV $i$ and the sum of the image acquisition time and the flight time within the sensing range assigned to UAV $i$, respectively.
%
The max function of the fourth term in the right side of (\ref{eq_fin}) is the processing time of $N^i$ images, which is equal to the longer one of the image processing time at the UAV $i$ or the total consumed time for image transmission from UAV $i$ to the ES and the image processing at the ES.
%
$T_{u,d}^{i}(N^i,N_u^i)$, $T_{d,e}^{i}(N^i,N_u^i)$, and $T_e^{i}(N^i,N_u^i)$ in the right side of (\ref{eq_fin}) are waiting times for UAV $i$.
%
As we mentioned before, if previous UAVs (UAV1,UAV2,$\cdots$, UAV${i-1}$) are still using the communication channel of the UD or the computational resource of the ES, UAV $i$ has to wait for a certain waiting time until all the transmission and processing tasks have been completed.
%
That is, $t_{fin}^{i-1}$, which is the time when processing $N^{i-1}$ images acquired by UAV $i-1$ is finished, is not affected by the operation of UAV $i$ and can be dealt as a constant value in the scheduling of UAV $i$.
%
Note that we assumed, since the size of output data obtained after processing at UAVs and the ES is quite small, transmission time of those output data is negligible.

$N_{MIN}^i$ in (\ref{eq2}) is the specific value of $N^i$ that satisfies the following condition:
%
\begin{align}
\argmin_{N^i,N_u^i} \quad& {t_{fin}^i(N^i,N_u^i)}\label{eq_min1}\\
s.t. \quad& {t_{fin}^i(N^i,N_u^i)}\geq{t_{fin}^{i-1}}({N_u^i}\in \mathbb{N}\mid 0\leq{N_u^i}\leq{N^i}). \label{eq_min2}
\end{align}
%
Here, suppose that ${N_u^i}$ is determined so as to minimize $t_{fin}^i(N^i,N_u^i)$ for a given $N^i$.
%
For such ${N_u^i}$, $N_{MIN}^i$ is the minimum integer among possible values of $N^i$ that satisfy Formula (\ref{eq_min2}).
%
By setting $N_{MIN}^i$ so, as long as ${N_u^i}$ is chosen so as to satisfy $0\leq{N_u^i}\leq{N^i}$, the optimal $N^i$ in (\ref{eq1}) can be determined among the possible values of $N^i$ that satisfy $U^i$(=$\frac{\eta^{i}}{{\Delta{t}}^i}) > 0$.

The solution of (\ref{eq1}) and (\ref{eq2}) is mentioned in datail in APPENDIX \ref{ape}.

\subsection{feature of proposed scheduling method}

Our proposed scheduling method is service-centric; since it has service-centric features as listed below.
\begin{itemize}
\item Robustness for the increase or decrease of number of UAVs: 
The number of UAVs may increase due to adding the new UAVs to the existing UAV group or decrease due to the breakdown of the UAVs.we deal with such cases since each UAV is scheduled sequentially one by one.
%
\item Applicability for the Heterogeneity of UAVs: 
There are individual differences in the flying speed and the processing speed of each UAV. In our proposed scheduling method, such individual differences can be considered.
%
\item Applicability for the various kinds of processing capacities of UAVs and ES: 
The computing powers of each UAV and ES depend greatly on the machine performance. In the proposed method, regardless of the machine performance, it is possible to efficiently utilize the computing capacity of each UAV and ES.
%
\item Feasibility for various types of geographical areas: 
There are many kinds of natural disasters occurred all over the world, such as earthquakes, hurricanes, forest fires, and floods. 
Our proposed scheduling method can be used in the various disaster-damaged areas.
\end{itemize}

\section{Basic Performance evaluation}\label{eva}

\subsection{Simulation scenario}
A basic simulation was performed to validate the proposed scheduling method.
%
We considered a surveillance scenario in which a rescue team uses the multi-UAV system illustrated in Figure \ref{model} to find missing people in an area where humans and ground vehicles cannot easily step into.
%
Although our scheduling method should be applicable for realistic scenarios, we assume a simple model illustrated in Figure \ref{model1.5}, in which sensing sections are placed on the one-dimensional line and their sizes are identical, to present the problem formulation.
%
Our simulation adopted the proposed scheduling method described in Section \ref{math} and performed every step of the system flow described in Section \ref{flow}.
%
We compared the proposed method with one of the existing methods: fixed method, which simply assigns a fixed number of sensing sections to each UAV uniformly \cite{chang2016}.
In the fixed method, $N_u^i$ is determined so that $t_{fin}^i$ is minimized.
また提案方式と固定方式ともに,UAV $i$は$N^i$枚の画像処理が終了すると即座に次の飛行を開始する.
%
In our basic evaluation, we adopts the following evaluation metric: the cumulative sum of utilities, defined by (\ref{ut}), against elapsed time.
ユーザはUAVの処理結果を取得するごとに効用を受け取ることができるものとし,その累積和を考える.

\subsection{Simulation parameters}

The parameters used in our simulation are listed in Table \ref{para_val}.
%
Considering the realistic specification of a recently commercialized UAV \cite{bebop2}, we set the flying speed of UAVs to 15 m/s.
%
%また図\ref{model1.5}におけるUAVの飛行開始位置から調査エリアの左端までの距離は災害発生地域や森林山岳地帯など人が立ち入れない地域での利用を想定し設定した.
The size of images was set to 100 kbytes, which corresponds to the one in the dataset called PASCAL VOC 2007 used in \cite{Ren2015}.
%
The consumed times for processing one image at UAVs and the ES are set corresponding to the consumed time for object recognition using GPU and CPU reported in \cite{Ren2015}, respectively.
またUAVの個体差を考慮して,各UAVの速度と単位時間あたりの画像処理枚数はそれぞれ平均値$\mu$の10\%を標準偏差$\sigma$とする正規分布から$\mu{\pm}2{\sigma}$の範囲でランダムに設定した.
%
The transmission rates of communication channel from UAVs to the UD and from the UD to the ES are set 100 Mbps, which is similar to the effective throughput of IEEE802.11n \cite{Li2013}.
%

\subsection{Results}
Figures \ref{utility} (a) and \ref{utility} (b) plots cumulative sum of utilities against elapsed time with $D$=200 and $D$=600, respectively.
We examined the fixed method with $N^i$ = 1, 10, 40, and 80.
いずれも時間の経過とともに単調増加しているが,$D$や$M$の値によって固定方式は優劣が変化する一方で,提案方式は常に値が最大となっている.
よってユーザが得られる効用の累積和の観点において固定方式よりも提案方式の方が優れていると言える.

\begin{figure*}[t]
\begin{center}
\includegraphics[width=16.0cm]{fig/cum_uti.png}
\caption{Cumulative sum of utilities}
\label{utility}
\end{center}
\end{figure*}

\begin{table}
  \begin{center}
    \caption{Simulation parameters}
    \label{para_val}
    \begin{tabular}{ll}
     \hline
Parameters & value \\ \hline
No. of UAVs ($U$) & 5\\
Average flying speed of UAVs & 15 m/s \\
\shortstack[l]{Distance between initial position \\and left-end of sensing block ($D$)} & \shortstack[l]{200,600m\\~}  \\ 
Size of each sensing section & 5 m  \\ 
Consumed time for acquiring one image & 2 s \\ 
Transmission rate of communication channel & 100 Mbps \\ 
Size of image  &  100 kbytes \\ 
\shortstack[l]{Average processing speed at UAVs \\in no. of images per unit time ($\overline{P_u}$)} & \shortstack[l]{$\frac{1}{1.83}$\\~} \\ 
\shortstack[l]{Processing speed at ES in no. of images \\per unit time ($P_e$)} & \shortstack[l]{$\frac{1}{0.198}$\\~} \\ \hline
    \end{tabular}
  \end{center}
\end{table}

\section{QoS Evaluation}
\subsection{Extension to two-dimensional model}\label{twodi}
In QoS Evaluation, the proposed scheduling method described in Section \ref{math} was applied directly to the two-dimensional model.
There are two assumptions in which the proposed scheduling method can be extended to the two-dimensional model shown in Figure \ref{twodimention} :

\begin{description}

\item[(1)]  Assuming fan-shaped sensing region, split the region so that each sensing section range is constant
\item[(2)] Each UAV executes sensing on the zigzag in order from the area closest to the center like \cite{Maza2007}
\end{description}

As shown in Figure \ref{twodimention}, assuming that the vertical width of each partition is $d$ and the central angle of the sensing region is $\theta$ [rad], the distance between each section when UAV moves in the direction opposite to the center of the region is $d$ and the distance between each section when moving on the circumference is $\frac{1}{2}d\theta$.
Therefore, setting $\theta=2$ [rad] so that all the distance between each section are equal as in Figure \ref{model1.5}, the proposed scheduling method described in Section \ref{math} was applied to the Figure \ref{twodimention}.

\begin{figure}[t]
\begin{center}
\includegraphics[width=8.0cm]{fig/twodim.png}
\caption{Sensing region of two-dimensional model}
\label{twodimention}
\end{center}
\end{figure}

\subsection{Evaluation metric}\label{compare}
In our QoS evaluation, we used the following three evaluation metrics.

\paragraph*{・two types of elapsed times}
The first elapsed time is `elapsed time from start time' for each image, which is the elapsed time since the first UAV starts flying until the result about each image is obtained by the UD.
%
This metric is important for the rescue team because they need to know the information about each sensing section as soon as possible to know whether missing people are there or not.
%
The second one is `elapsed time after acquired' for each image, which is the elapsed time since an image is acquired at the corresponding sensing section until the result about the image is obtained by the UD.
%
This metric is also valuable for the rescue team because it indicates the freshness of the information about each sensing section; the less updated information, the less reliable for them in searching missing people.

\paragraph*{・ Number of images whose elapsed time after acquired is less than a predetermined value at a certain time}
We also adopts the metric combining the above two types of elapsed times: Number of images whose elapsed time after acquired is less than a predetermined value when elapsed time from start time is a certain value.
This metric is important for the rescue team because they need to get the fresh information while obtaining as many pieces of information as possible for a certain period.


\paragraph*{・Cumulative sum of values}
The third metric is cumulative sum of values at each time.
In \cite{NOMURA2001}, the value function of the acquired image $j$ at a elapsed time from start time $t$ was given as:
\begin{align}
V_j=2^{-\frac{t}{T_{half}}}, \label{eq_value}
\end{align}
where $T_{half}$ means the half-life of values, which is set considering the time it takes for the rescue team to be sent out. 
%$T_{half}$は半減期を表す定数である.
式(\ref{eq_value})は,取得画像$j$の価値が画像が撮影された時刻に1を取り時間経過とともに指数減衰することを表す.
ユーザは取得結果$j$を受け取るごとに価値$V_j$を受け取ることができるものとし,その累積和を考える.

\subsection{result}

\begin{figure}[t]
\begin{center}
\includegraphics[width=8cm]{fig/elapsedtime.png}
\caption{Two types of elapsed times}
\label{elapsed}
\end{center}
\end{figure}

\begin{figure*}[t]
\begin{center}
\includegraphics[width=18.0cm]{fig/noimage.png}
\caption{Number of images whose elapsed time after acquired is less than 120 [s]  at 600 [s]}
\label{totalnumber}
\end{center}
\end{figure*}

\begin{figure*}[t]
\begin{center}
\includegraphics[width=18.0cm]{fig/totalvalue.png}
\caption{Cumulative sum of values}
\label{totalvalue}
\end{center}
\end{figure*}

Figure \ref{elapsed} (a) plots the elapsed time from the start time for acquired images against each area id.
In Fig. \ref{elapsed} (a), as the area id increases, the elapsed time from start time monotonically increases in all the methods.
In the fixed method, as the area id increased, $N^i$ when the elapsed time from the start time was the shortest at each area id was larger.
The proposed method worked between the fixed method with $N^i=10$ and $N^i=40$._
%
On the other hand, Figure \ref{elapsed} (b) plots the elapsed time after acquired for each image.
The elapsed time after acquired decreases with some regular pattern in all the methods.
However, in the fixed method, as $N^i$ was set smaller, the elapsed time after acquired became shorter.
The proposed method worked between the fixed method with $N^i=10$ and $N^i=40$.
%
From the overall observation shown above, in the fixed method, $N^i=10$ and $N^i=40$ are reasonable; the elapsed times from start time of them were between the best and the third best, while their elapsed times after acquired were much shorter than that of $N^i=80$.
%
The proposed method worked between the fixed method with $N^i=10$ and $N^i=40$, which suggests that it enables us to automatically achieve the most reasonable $N^i$.

Figure \ref{totalnumber} (a), \ref{totalnumber} (b), and \ref{totalnumber} (c) plot number of images whose elapsed time after acquired is less than 120 [s] at 600 [s] against the processing speed at ES in no. of images per unit time, the number of UAVs, and the distance between initial position and center of sensing block where area id is 1, respectively.
また各値は三回のシミュレーションの平均値である.
図\ref{totalnumber}(a)上の破線は$P_e$が表\ref{para_val}に記した値の際の$\frac{P_e}{\overline{P_u}}$を表す.
In the left side of the \ref{totalnumber} (a), the processing speed at ES is sufficiently lower than the average processing speed at UAVs, which corresponds to the case where processing is performed only at UAV without ES.
In the right side of the \ref{totalnumber} (a), the average processing speed at UAV is sufficiently lower than the processing speed at ES, which corresponds to the case where processing is performed only at ES without UAV.

In Figure \ref{totalnumber}(a), in the proposed method, the number of images is continuously large irrespective of the the processing speed at ES: $N^i=40$の固定方式にのみ$\frac{P_e}{\overline{P_u}}$が$10^1$から$10^3$の範囲で劣っているが,$10^{-3}$から$10^{1}$の範囲では提案方式が最も優れている.
This indicates that the proposed method efficiently utilizes the computing capacity of each UAV and ES.
%よってESの計算能力によらず安定して値が大きく,各UAVやESの計算能力を効率的に利用出来ているが確認できる.
同様に図\ref{totalnumber}(b)(c)においても提案方式は固定方式と比較すると$U$や$D$の値による変化が小さく,平均して高い値を取っている.
Therefore, the proposed method is better than the fixed method since it is hardly affected by the parameter change.

Figure \ref{totalvalue} plots the cumulative sum of values against the elapsed time from start time.
The half-life of Figure \ref{totalvalue}(a), \ref{totalvalue}(b), and \ref{totalvalue}(c) is 30, 60, and 120 [s], respectively..
In these figures, the cumulative sum of values monotonically increases in all the methods.
固定方式においては半減期の値によって各方式の優劣が変化する.
しかし提案方式は半減期の違いによる影響が小さく,平均して値が大きくなっている.
The proposed method is better than the fixed method since it is hardly affected by the parameter change.

\section{CONCLUSION}
This paper proposed a scheduling method of multi-UAV search system that considers processing time of the acquired image data and data transfer time in areas where humans and ground vehicles cannot easily step into like disaster-damaged areas.
In this paper, we first showed the system model, which consists of a user device, multiple UAVs, and an edge server, and mentioned the sytem flow. 
We then presented the problem formulation of the proposed scheduling method that maximizes the user’s utility, which is calculated from the efficiency of obtaining results from analyzed data and the interval of obtaining the results, by using the one-dimensional model where sensing sections are placed on the one-dimensional line and their sizes are identical.
A simulation was performed to verify that the proposed method works well to ensure the freshness of individual obtained piece of information while delivering as many pieces of information as possible for a certain period.
The results indicated that the proposed method works better than the conventional fixed methods, in terms of two metrics: i) elapsed time from start time and ii) elapsed time after acquired for each image.
For a more practical evaluation, in our future work, we will include prototype implementation and experiment.

%\bibitem{Pitre2012} R. R. Pitre, X. R. Li, R. Delbalzo, “UAV Route Planning for Joint
%Search and Track Missions-An Information-Value Ap-proach”,  \emph{IEEE
%Trans. Aerospace and Electronic Systems}, vol. 48, no.3, pp.2551 -2565, 2012

% if have a single appendix:
%\appendix[Proof of the Zonklar Equations]
% or
%\appendix  % for no appendix heading
% do not use \section anymore after \appendix, only \section*
% is possibly needed

% use appendices with more than one appendix
% then use \section to start each appendix
% you must declare a \section before using any
% \subsection or using \label (\appendices by itself
% starts a section numbered zero.)
%


\appendices
\section{Solution of the problem formulation}\label{ape}
It takes long time to solve (4) and (5) due to their computational complexities and the calculation for scheduling could be non-negligible overhead in the system; UAVs have to wait to start flying until their schedules have been determined.
Therefore, to simplify the calculation of (\ref{eq1}) and (\ref{eq2}), our scheduling method first supposes that all the waiting times for UAV $i$ are equal to zero and obtains an approximated solution, $N_{th}^i$.
Then, by searching locally around $N_{th}^i$, our scheduling method considers all the waiting times in (\ref{eq1}) and (\ref{eq2}) and finds out the local optimal solution, $N_{l}^i$.
The following part explains the way of finding out $N_{th}^i$.
%
First, we set all the waiting times to zero.
%
Then, we replace the second and fourth terms in (\ref{eq_fin}) with $2F$ and $M$, respectively.
%
Then, by substituting ${t_{fin}^i}$ in (\ref{eq_fin}) to (\ref{ut}), we obtain $\frac{\eta^i}{{\Delta{t}}^i}$ as below:
%
\begin{align}
\frac{\eta^{i}}{{\Delta{t}}^i}
=&\frac{N^i}{N^i({T_g}+T_{ng})+2F+M}\times\nonumber\\
&\frac{1}{N^i({T_g}+T_{ng})+2F+M+t_{start}^i-t_{fin}^{i-1}}\label{eq4}
\end{align}
%
where, by regarding $N^i$ as a constant, only $M$ is a variable and the value of $\frac{\eta^{i}}{{\Delta{t}}^i}$ varies dependently on the values of $N_u^i$ and $N_e^i$.
%
Since $\frac{\eta^{i}}{{\Delta{t}}^i}$ is always positive,  $M$ takes the minimum when $\frac{\eta^{i}}{{\Delta{t}}^i}$ takes the maximum. 
%
Although $N^i$, $N_u^i$, and $N_e^i$ are integers, here we deal with them as real numbers.
%
Considering $N^i=N_u^i + N_e^i$, on the assumption that $\frac{N_u^i}{P_u^i}$ equals to $\frac{N_e^i}{\mu_{u,d}}+\frac{N_e^i}{\mu_{d,e}}+\frac{N_e^i}{P_e}$, we can obtain the value of $N_u^i$ and $N_e^i$ that satisfy $0\leq{N_u^i}$ and ${N_e^i}\leq{N^i}$ as follows:
%
\begin{align}
N_u^i=\frac{P_u^i(\mu_{d,e}P_e+\mu_{u,d}P_e+\mu_{u,d}\mu_{d,e})}{\mu_{u,d}\mu_{d,e}P_e+P_u^i(\mu_{d,e}P_e+\mu_{u,d}P_e+\mu_{u,d}\mu_{d,e})}N^i\label{eq5}\\
N_e^i=\frac{\mu_{u,d}\mu_{d,e}P_e}{\mu_{u,d}\mu_{d,e}P_e+P_u^i(\mu_{d,e}P_e+\mu_{u,d}P_e+\mu_{u,d}\mu_{d,e})}N^i\label{eq6}
\end{align}
%
When $M$ takes the minimum value, $\frac{N_u^i}{P_u^i}$ is theoretically equal to $\frac{N_e^i}{\mu_{u,d}}+\frac{N_e^i}{\mu_{d,e}}+\frac{N_e^i}{P_e}$, while $N_u^i$ and $N_e^i$ become (\ref{eq5}) and (\ref{eq6}), respectively.
%
As $N^i$ becomes larger, $\frac{N_u^i}{P_u^i}$ is closer to $\frac{N_e^i}{\mu_{u,d}}+\frac{N_e^i}{\mu_{d,e}}+\frac{N_e^i}{P_e}$.
%
Thus, $\frac{N_u^i}{P_u^i}=\frac{N_e^i}{\mu_{u,d}}+\frac{N_e^i}{\mu_{d,e}}+\frac{N_e^i}{P_e}$ is established.\\
%
As a result, $\frac{\eta^{i}}{{\Delta{t}}^i}$ in (\ref{eq4}) is given as:
%
\begin{align}
&\hspace{-0.2cm}\frac{N^i}{R^2{(N^i)}^2+(4F+t_{start}^i-t_{fin}^{i-1})R{N^i}+2F(2F+t_{start}^i\hspace{-1mm}-t_{fin}^{i-1})}\hspace{1cm}\label{eq_ec}\\
&\hspace{-0.6cm}\biggl(R=T_g+T_{ng}+\frac{\mu_{d,e}P_e+\mu_{u,d}P_e+\mu_{u,d}\mu_{d,e}}{\mu_{u,d}\mu_{d,e}P_e+P_u^i(\mu_{d,e}P_e+\mu_{u,d}P_e+\mu_{u,d}\mu_{d,e})}\biggr)\nonumber
\end{align}
%
To sketch (\ref{eq_ec}), by differentiating it by $N^i$, we obtain:
%
\begin{align}
\hspace{-4mm} \frac{-{R^2{(N^i)}^2}+2F(2F+t_{start}^i\hspace{-1mm}-t_{fin}^{i-1})}{\{R^2{(N^i)}^2+(4F+t_{start}^i\hspace{-1mm}-t_{fin}^{i-1})R{N^i}+2F(2F+t_{start}^i\hspace{-1mm}-t_{fin}^{i-1})\}^2}
\end{align}
%
When $t_{start}^i \leq{t_{fin}^{i-1}-2F}$, (\ref{eq_ec}) decreases monotonically as the value of $N^i$ increases and takes the maximum when $N_{th}^i$ is $N_{MIN}^i$.
%
When $t_{start}^i >{t_{fin}^{i-1}-2F}$, $\frac{\eta^{i}}{{\Delta{t}}^i}$ is a convex function taking the maximum when $N^i$ is $\frac{\sqrt{2F(2F+t_{start}^i-t_{fin}^{i-1})}}{R}$.
%
Note that $N^i$ is chosen so that the denominator of (\ref{eq_ec}) does not become zero.
%
Through the above procedures, we can obtain $N_{th}^i$ as follows:
%
\begin{align}
 \hspace{-1.5mm} N_{th}^i= \begin{cases}
    \frac{\sqrt{2F(2F+t_{start}^i-t_{fin}^{i-1})}}{R} & ({N_{MIN}^i}\leq{\frac{\sqrt{2F(2F+t_{start}^i-t_{fin}^{i-1})}}{R}}) \\
    N_{MIN}^i& (\frac{\sqrt{2F(2F+t_{start}^i-t_{fin}^{i-1})}}{R}<{N_{MIN}^i})
  \end{cases}
\end{align}

In finding out the local optimal solution, $N_{l}^i$, the range of local searching in the proposed scheduling method was set enough wide so that it can ensure that the local optimal solution equals to the true optimal solution.
%
In addition, we assumed that the consumed time for finding out the local optimal solution is negligible; since the complexity of the scheduling algorithm is quite simple, the calculation can be finished in advance while the UAV is flying in the previous round.




% Can use something like this to put references on a page
% by themselves when using endfloat and the captionsoff option.
\ifCLASSOPTIONcaptionsoff
  \newpage
\fi

% trigger a \newpage just before the given reference
% number - used to balance the columns on the last page
% adjust value as needed - may need to be readjusted if
% the document is modified later
%\IEEEtriggeratref{8}
% The "triggered" command can be changed if desired:
%\IEEEtriggercmd{\enlargethispage{-5in}}

% references section

% can use a bibliography generated by BibTeX as a .bbl file
% BibTeX documentation can be easily obtained at:
% http://mirror.ctan.org/biblio/bibtex/contrib/doc/
% The IEEEtran BibTeX style support page is at:
% http://www.michaelshell.org/tex/ieeetran/bibtex/
%\bibliographystyle{IEEEtran}
% argument is your BibTeX string definitions and bibliography database(s)
%\bibliography{IEEEabrv,../bib/paper}
%
% <OR> manually copy in the resultant .bbl file
% set second argument of \begin to the number of references
% (used to reserve space for the reference number labels box)
\begin{thebibliography}{1}
\bibitem{CRED2016}Debarati Guha-Sapir, Philippe Hoyois, Pasacline Wallemacq and Regina Below,``Annual Disaster Statistical Review 2016,"\emph{Centre for Research on
the Epidemiology of Disasters},2016 
\bibitem{disaster2011} Japan Ministry of Defense, ``Lessons from the Great East Japan Earthquake," \url{http://www.mod.go.jp/e/publ/w_paper/pdf/2012/30_Part3_Chapter1_Sec3.pdf}, 2012
\bibitem{Andre2014} T. Andre, K. Hummel, A. Schoellig, E. Yanmaz, M. Asadpour, C. Bettstetter, P. Grippa, H. Hellwagner, S. Sand and S. Zhang, ``User devicelication-driven design of aerial communication networks, " \emph{IEEE Commun. Mag.}, vol. 52, no. 5, pp. 129-137, 2014
\bibitem{Erdelj2016} M. Erdelj, and N. Enrico, ``UAV-assisted disaster management: User devicelications and open issues, " \emph{2016 International Conference on Computing, Networking and Communications (ICNC)},pp1-5, 2016
\bibitem{Felice2014} M. Di Felice, A. Trotta, L. Bedogni, K. R. Chowdhury and L. Bononi, ``Self-organizing aerial mesh networks for emergency communication, " \emph{Personal, Indoor, and Mobile Radio Communication (PIMRC), IEEE 25th Annual International Symposium on}, pp. 1631-1636, 2014
\bibitem{japan2011}Armand Vervaeck and James Daniell,``Japan Tohoku tsunami and earthquake : The death toll is climbing again!, " \url{https://earthquake-report.com/2011/08/04/japan-tsunami-following-up-the-aftermath-part-16-june/}, 2011
\bibitem{Lanillos2014} P. Lanillos, S. K. Gan, E. Besada-Portas, G. Pajares and S. Sukkarieh, ``Multi-UAV target search using decentralized gradient-based negotiation with expected observation,"\emph{Information Sciences}, vol. 282, pp. 92-110, 2014.
\bibitem{Maza2007} I. Maza, A. Ollero, ``Multiple UAV cooperative searching operation
using polygon area decomposition and efficient coverage algorithms, "\emph{Distributed Autonomous Robotic Sys-tems}, pp. 221-230, 2007.
\bibitem{Meng2014} W. Meng, Z. R. He, R. Teo, L. xie, ``Decentralized Search, Tasking and Tracking Using Multiple Fixed-Wing Miniature UAVs," \emph{11th IEEE
International Conference on Control \& Automation (ICCA)}, pp. 1345-1350, 2014 
\bibitem{chang2016} C. Zhao, M. Zhu and H. Liang, ``The Sustainable Tracking Strategy of Moving Target by Multi-UAVs in an Uncertain Environment, " \emph{2016 IEEE/CSAA International Conference on Aircraft Utility Systems (AUS)}, pp. 20-25, 2016
\bibitem{Mirzaei2011} M. Mirzaei, F. Sharifi, B. W. Gordon, C. A. Rabbath and  Y. M.Zhang, ``Cooperative multi-vehicle search and coverage problem in uncertain environments, "\emph{in Proceedings of the 50th IEEE Conference on Decision and Control and European Control Conference (CDCECC)}, pp. 4140-4145, 2011
%
\bibitem{Bamburry2015} D. Bamburry, ``Drones: Designed for product delivery," \emph{Design Management Review}, vol. 26, no. 1, pp. 40-48, 2015
\bibitem{May2015}K. May,``Drones to deliver medicine and food? Drones for disaster relief? Why not?,"\url{https://ideas.ted.com/6-ways-drones-can-be-used-for-good/}, 2013
\bibitem{Wada2015}A. Wada, T. Yamashita, M. Maruyama, T. Arai, H. Adachi and H. Tsuji, ``A surveillance system using small unmanned aerial vehicle (UAV) related technologies,’’\emph{NEC Technical Journal},vol.8, no. 1, p68-72, 2015
\bibitem{Bekmezci2013} I. Bekmezci, O. K. Sahingoz, and S. Temel, “Flying Ad-Hoc Networks (FANETs): a survey,” \emph{Ad Hoc Networks}, vol. 11, no. 3, pp. 1254-1270, 2013
\bibitem{Garcia2016} J S\'anchez-Garc\'ia, JM Garc\'ia-Campos, SL Toral, DG Reina and F Barrero, ``An intelligent strategy for tactical movements of UAVs in disaster scenarios,’’\emph{
International Journal of Distributed Sensor Networks}, vol.12, no. 3, 2016
\bibitem{Hayat 2017}S. Hayat, E. Yanmaz, T. X. Brown, and C. Bettstetter,``Multi-Objective UAV Path Planning for Search and Rescue,” \emph{ICRA Singapore}, 2017 
%
%
\bibitem{Ouahouah2017}S. Ouahouah , T. Taleb, J. Song and C. Benzaid, ``Efficient offloading mechanism for UAVs-based value added services," \emph{IEEE International Conference on Communications (ICC)}, 2017
\bibitem{Valentino2018} R. Valentino, W.-S. Jung and Y.-B. Ko,``Opportunistic Computational Offloading System for Clusters of Drones,” \emph{International Conference on Advanced Communications Technology(ICACT)}, 2018 
%\bibitem{Nigam2014} N. Nigam, ``The Multiple Unmanned Air Vehicle Persistent Surveillance Problem: A Review, \emph{Machines}, Vol.2, Issue 1, pp. 13-72, 2014
%\bibitem{Loke2015} S. W. Loke, "The internet of flying-things: Opportunities and challenges with airborne fog computing and mobile cloud in the clouds, " \emph{arXiv preprint arXiv:1507.04492}, 2015
%\bibitem{Jeong2016} S. Jeong, O. Simeone, and J. Kang, "Mobile edge computing via a UAVmounted cloudlet: Optimal bit allocation and path planning, "\emph{IEEE Transactions on Vehicular Technology,to appear}, 2017
\bibitem{Motlagh2017}N. H. Motlagh, M. Bagaa, and T. Taleb, ``Uav-based iot platform: A crowd surveillance use case,\emph{IEEE Communications Magazine}, vol.55, no.2, pp. 128-134, 2017
%\bibitem{Mohamed2017}M.-A. Messous, H. Sedjelmaci, N. Houari, S.-M. Senouci, ``Computation offloading game for an uav network in mobile edge computing,\emph{IEEE International Conference on Communications (ICC)}, pp. 1-6, 2017
\bibitem{Messous2017} M.-A. Messous1, A. Arfaoui, A. Alioua, S. -M. Senouci,``A Sequential Game Approach for Computation-Offloading in an UAV Network,’’\emph{IEEE Global Communications Conference}, 2017 
\bibitem{bebop2} Parrot SA, ``PARROT BEBOP2 SO LIGHT YOU CAN TAKE IT
ANYWHERE TO FILM IN FULL HD, "
\url{http://www.parrot.com/usa/products/bebop2/}, 2016
%\bibitem{UPF614496} ``Lithium Ion UPF614496,'' Panasonic, \url{http://www.panamar.it/images/date_sheets_UPF-614496.pdf}, Jan.20, 2016

%\bibitem{Traub2016} L.W. Traub,  ``Calculation of Constant Power Lithium Battery Discharge Curves,'' \emph{Batteries}, vol2, no2, 2016
\bibitem{Ren2015} S. Ren, K. He, R. Girshick, J. Sun, ``Faster R-CNN: Towards
real-time object detection with region proposal networks,'' \emph{In Advances in neural information processing systems}, pp. 91-99, 2015
\bibitem{Li2013} J. Li, Y. Fan, H. Chen, K. Xu, Y. Dai, F. Yin, Y. Ji,  ``Radio-over-fiber-based distributed antenna systems supporting IEEE 802.11 N/AC standards,'' \emph{Optical Communications and Networks (ICOCN), 2013 12th International Conference on. IEEE}, pp. 1-4, 2013
\bibitem{NOMURA2001}K. NOMURA, K. YAMORI, E. TAKAHASHI, T. MIYOSHI, and Y. TANAKA, ``Waiting time versus utility to download images.,"  \emph{4th Asia Pacific-Symposium on Information and Telecommunication Technologies}, pp. 128-132, 2001 

\end{thebibliography}


\EOD


\end{document}


